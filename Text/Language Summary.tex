
\begin{description}
	\item[S (CHERI-RISC-V Assembly)] The ground language as provided to Clang for assembly and linking.
	\item[RV (CHERI-RISC-V)] A language that maps directly to CHERI-RISC-V (pseudo-)instructions.
	\item[FL (Frame Locations)] A language that introduces frame locations, i.e., memory locations relative to the frame capability `cfp`.
	\item[FO (Frame Operands)] A language that introduces flexible operands in instructions, i.e., instructions that can take frame locations in all operand positions.
	\item[BB (Basic Blocks)] A language that groups effects into blocks of effects where blocks can only be entered at a single entry point and exited at a single exit point.
	\item[PR (Predicates)] A language that introduces predicates in branches.
	\item[CD (Conditionals)] A language that introduces conditionals in effects and predicates, thereby abstracting over blocks (and jumps).
	\item[AL (Abstract Locations)] A language that introduces abstract locations, i.e., locations whose physical locations are not specified by the programmer.
	\item[CC (Calling Convention)] A language that introduces parameter passing and enforces the low-level Glyco calling convention.
	\item[CA (Canonical Assignments)] A language that groups all effects that write to a location under one canonical assignment effect.
	\item[CV (Computed Values)] A language that allows computation to be attached to an assigned value.
	\item[DF (Definitions)] A language that introduces definitions with function-wide namespacing.
	\item[LS (Lexical Scopes)] A language that introduces lexical scopes of definitions
	\item[EX (Expressions)] A language that introduces expression semantics for values, thereby abstracting over computation effects.
\end{description}