\documentclass[main.tex]{subfiles}
\begin{document}
\onlyinsubfile{\mainmatter{}}

\chapter{Sealed Objects}
The previous chapter discussed a first extension of the Glyco compiler, namely a secure calling convention. This chapter treats a second extension, a feature we call \emph{sealed objects}.

Sealed objects depend on two new features; we begin by defining lambdas in \cref{sct:lambda} and named \& nominal types in \cref{sct:named-ty} before defining the semantics of objects and methods in \cref{sct:obj-meth}. We then list in \cref{sct:obj-sec} a few security properties afforded by sealed objects. We finish this chapter by outlining the changes to the compiler in \cref{sct:obj-changes} and evaluating sealed objects in \cref{sct:obj-eval}.

This chapter discusses the feature set and languages of Glyco 1.0.\footnote{The source code is available at \url{https://tsarouhas.eu/glyco/1.0/}.} A full language reference can be found in \cref{ch:grammar}.

\section{Lambdas} \label{sct:lambda}
A first addition to Glyco is the \textbf{lambda}, i.e., an anonymous function, in a new \g{il} called \textbf{Λ} (Lambdas) above EX. Lambdas allow the programmer to define functions (\texttt{λ} values) at the point of use and to pass them around as values. For example, the following Λ program defines a lambda that computes the sum of its two parameters and immediately applies it on $1080$ and $-80$.
\ilfile{Programs/sum.l}

The \g{nanopass} from Λ to EX extracts the functions defined by \texttt{λ} values into the global scope (with a auto-generated name) and replaces the \texttt{λ} values by code capabilities to those functions. The above program is thus \lowered{} to the EX program
\ilfile{Programs/sum.ex}

\section{Named \& Nominal Types} \label{sct:named-ty}
A second feature is the ability to name types, by either defining a nominal type or a structural type, in a new \g{il} \textbf{NT} (Named Types) above Λ. A value in NT can be
\begin{itemize}[nosep]
	\item an (8-bit) byte, an \iil/u8()/;
	\item a (32-bit) signed integer, an \iil/s32()/;
	\item a capability to a vector of \iil/T/s, a \iil/cap(vector(T, sealed: S))/;
	\item a capability to a record, a \iil/cap(record((name, type) …, sealed: S))/;
	\item a capability to a function, a \iil/cap(function(takes: p …, returns: r))/;
	\item a \g{sealcap}, a \iil/cap(seal(sealed: S))/; or
	\item a type named \iil/T/, spelled \iil/named(T)/
\end{itemize}
where \iil/S/ is \iil/true/ if the capability is sealed and \iil/false/ otherwise.

Types can be named using a \texttt{letType} value in one of two ways. A \textbf{structural type definition} creates a \textbf{named structural type}, a type that is equivalent to the type it's defined to be. A \textbf{nominal type definition} creates a \textbf{nominal type}, a type which is not equivalent to any other type than itself.

For instance, the following NT program using structural types is valid:
% TODO: Add NT example program.

The following program using nominal types is also valid:
% TODO: Add NT example program.

However, this program is not valid since the nominal type cannot be converted to a \iil/s32()/ even though it is defined using \iil/s32()/:
% TODO: Add NT example program.

The \g{nanopass} from NT to Λ replaces named types by their Λ equivalents and checks if the nominal typing rules hold.

\section{Objects \& Methods} \label{sct:obj-meth}
A \textbf{sealed object} (or simply \textbf{object} in this chapter) is an encapsulated record on which a predefined set of functions can operate. These functions are the object's \textbf{methods}.

An object belongs to an \textbf{object type}, which determines the record type and methods of all objects of that type. Sealed objects in Glyco are in this respect similar to objects in object-oriented programming languages such as Java. Object types however do not support inheritance and record fields cannot be made visible outside of methods.

Object types can be defined using an \textbf{object type definition} in a \texttt{letType} value.

The following OB program implements a \texttt{Counter} object type, creates a counter with an initial value of 32, increases the count three times, and evaluates to the counter's final value (35):
\ilfile{Programs/counter.ob}

An object type consists of three parts:
\begin{itemize}
    \item A \textbf{name}, which is \texttt{Counter} in the example.
\end{itemize}

An object in Glyco is represented as a sealed capability to a record.

\subsection{Object Types are Objects}
% TODO

\section{Object State Encapsulation} \label{sct:obj-sec}
% TODO

\section{Changes to Glyco} \label{sct:obj-changes}
% TODO

\section{Evaluation} \label{sct:obj-eval}
% TODO

% \biblio{} TODO: Uncomment after adding a citation in chapter.
\onlyinsubfile{\glsaddall\printglossaries}
\end{document}
