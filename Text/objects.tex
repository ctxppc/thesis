\documentclass[main.tex]{subfiles}
\begin{document}
\onlyinsubfile{\mainmatter{}}

\chapter{Sealed Objects}
The previous chapter discussed a first extension of the Glyco compiler, namely a secure calling convention. This chapter treats a second extension, a feature we call \emph{sealed objects}.

Sealed objects depend on two new features; we begin by defining lambdas in \cref{sct:lambda} and named \& nominal types in \cref{sct:named-ty} before defining the semantics of objects and methods in \cref{sct:obj-meth}. We then list in \cref{sct:obj-sec} a few security properties afforded by sealed objects. We finish this chapter by outlining the changes to the compiler in \cref{sct:obj-changes} and evaluating sealed objects in \cref{sct:obj-eval}.

This chapter discusses the feature set and languages of Glyco 1.0.\footnote{The source code is available at \url{https://tsarouhas.eu/glyco/1.0/}.} A full language reference can be found in \cref{ch:grammar}.

\section{Lambdas} \label{sct:lambda}
A first new addition to Glyco is the \textbf{lambda}, which is an anonymous function, in a new \g{il} called \textbf{Λ} (Lambdas). Lambdas allow the programmer to define functions at the point of use and to pass them around as values. For example, the following Λ program defines a lambda that computes the sum of its two parameters and immediately applies it on $1080$ and $-80$.
\lstinputlisting{Programs/sum.l}

\section{Named \& Nominal Types} \label{sct:named-ty}
% TODO

\section{Objects \& Methods} \label{sct:obj-meth}
A \textbf{sealed object} (or simply \textbf{object} in this chapter) is an encapsulated record on which a predefined set of functions can operate. These functions are the object's \textbf{methods}.

An object belongs to an \textbf{object type}, which determines the record type and methods of all objects of that type. Sealed objects in Glyco are in this respect similar to objects in object-oriented programming languages such as Java. Object types however do not support inheritance and record fields cannot be made visible outside of methods.

The highest \g{il} in Glyco is called \textbf{OB} (Objects), a language similar to EX but with support for named types, anonymous functions, and sealed objects.

The following OB program implements a \texttt{Counter} object type, creates a counter with an initial value of 32, increases the count three times, and evaluates to the counter's final value (35):
\lstinputlisting{Programs/counter.ob}

An object type consists of three parts:
\begin{itemize}
    \item A \textbf{name}, which is \texttt{Counter} in the example.
\end{itemize}

An object in Glyco is represented as a sealed capability to a record.

\subsection{Object Types are Objects}
% TODO

\section{Object State Encapsulation} \label{sct:obj-sec}
% TODO

\section{Changes to Glyco} \label{sct:obj-changes}
% TODO

\section{Evaluation} \label{sct:obj-eval}
% TODO

% \biblio{} TODO: Uncomment after adding a citation in chapter.
\onlyinsubfile{\glsaddall\printglossaries}
\end{document}
