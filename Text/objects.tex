\documentclass[main.tex]{subfiles}
\begin{document}
\onlyinsubfile{\mainmatter{}}

\chapter{Sealed Objects}
The previous chapter discussed a first extension of the Glyco compiler, namely a secure calling convention. This chapter treats a second extension, a feature we call \emph{sealed objects}.

A \textbf{sealed object} (or simply \textbf{object} in this chapter) is an encapsulated record on which a predefined set of functions can operate. These functions are the object's \textbf{methods}. An object belongs to an \textbf{object type}, which determines the record type and methods of all objects of that type. Sealed objects in Glyco are in this respect similar to objects in object-oriented programming languages such as Java. Object types however do not support inheritance and record fields cannot be made visible outside of methods.

We begin this chapter in \cref{sct:secobj} we list a few security properties afforded by sealed objects. We then discuss the semantics of objects and methods in \cref{sct:objmeth} and of object types in \cref{sct:objty}. We finish this chapter by outlining the changes to the compiler in \cref{sct:obj-changes} and evaluating sealed objects in \cref{sct:obj-eval}.

This chapter discusses the feature set and languages of Glyco 1.0.\footnote{The source code is available at \url{https://tsarouhas.eu/glyco/1.0/}.} A full language reference can be found in \cref{ch:grammar}.

\section{Secure Objects} \label{sct:secobj}
% TODO

\section{Objects \& Methods} \label{sct:objmeth}
% TODO

\section{Object Types} \label{sct:objty}
% TODO

\section{Changes to Glyco} \label{sct:obj-changes}
% TODO

\section{Evaluation} \label{sct:obj-eval}
% TODO

% \biblio{} TODO: Uncomment after adding a citation in chapter.
\onlyinsubfile{\glsaddall\printglossaries}
\end{document}
