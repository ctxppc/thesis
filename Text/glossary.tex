\newglossaryentry{sealing}{
	name=sealing,
	description={the act of setting the object type of an unsealed capability to either the address of a \sealcap{} or to a special value indicating that it's a \sentry{}, so that it can no longer be modified or dereferenced until it\'s \glslink{unsealing}{unsealed} again}
}

\newglossaryentry{unsealing}{
	name=unsealing,
	description={the act of setting the object type of a sealed capability to $-1$ so that it's no longer sealed and thus can be used normally}
}

\newglossaryentry{sentry}{
	name=sentry capability,
	plural=sentry capabilities,
	description={a sealed entry capability, a sealed capability created with the \texttt{CSealEntry} instruction which can only be unsealed by jumping to its address using the \texttt{CJALR} instruction}
}
\newcommand{\sentry}{\gls{sentry}}

\newglossaryentry{sealcap}{
	name=seal capability,
	plural=seal capabilities,
	description={a capability with the \emph{Permit seal} permission which can be used with the \texttt{CSeal} instruction to seal an unsealed capability with the seal capability's address}
}
\newcommand{\sealcap}{\gls{sealcap}}

\newglossaryentry{unsealcap}{
	name=unseal capability,
	plural=unseal capabilities,
	description={a capability with the \emph{Permit unseal} permission which can be used with the \texttt{CUnseal} instruction to unseal a capability sealed with the unseal capability's address}
}
\newcommand{\unsealcap}{\gls{unsealcap}}

\newdualentry{il}{IL}{intermediate language}{a language produced or parsed by a \gls{nanopass} as part of the compiler's routine}
\newcommand{\il}{\gls{il}}
\newcommand{\ils}{\glspl{il}}

\newglossaryentry{lowering}{
	name=lowering,
	description={the act of transforming a higher-level to a lower-level intermediate language through one or more nanopasses}
}
\newcommand{\lowered}{\glslink{lowering}{lowered}}

\newglossaryentry{nanopass}{
	name=nanopass,
	plural=nanopasses,
	description={a function that transforms a program in one \gls{il} to a program in another \gls{il}}
}

\newglossaryentry{lowerlang}{
	name=lower language,
	description=the \il{} output by a single \lowering{} of a program in the \il{} currently being discussed
}

\newglossaryentry{cc}{
	name=calling convention,
	description={a set of rules governing the invocation of procedures on a specific platform dealing with matters such as parameter \& result passing, register availability, call frame structure, jumping to the callee, and returning to the caller}
}
\newcommand{\cc}{\gls{cc}}
\newcommand{\ccs}{\glspl{cc}}

\newdualentry{gccc}{GCCC}{Glyco Conventional Calling Convention}{a \cc{} implemented in Glyco that is similar to a RISC-V \cc{}}

\newdualentry{ghscc}{GHSCC}{Glyco Heap-based Secure Calling Convention}{a \cc{} implemented in Glyco that ensures \gls{lse} by securely allocating call frames on the heap and sealing return \& frame capabilities with a unique seal before invocation}

\newglossaryentry{lse}{
	name=local state encapsulation,
	description={a security property guaranteeing that a procedure's state cannot be accessed by adversarial code in the same address space}
}

\newglossaryentry{wbcf}{
	name=well-bracketed control flow,
	description={a security property guaranteeing that a procedure can only invoke other procedures, return to its caller, or diverge}
}

\newglossaryentry{retcap}{
	name=return capability,
	plural=return capabilities,
	description={a capability containing a return address allowing a procedure to return to its caller}
}

\newglossaryentry{userp}{
	name=user program,
	description={a program not provided by the compiler or operating system, normally contrasted with the \g{rt}}
}

\newglossaryentry{rt}{
	name=runtime,
	description={compiler-provided software that runs at program startup, initialises data structures that are needed as part of the operation of the user program, and provides routines with privileges normally not afforded to \gs{userp}}
}

\newglossaryentry{rtrt}{
	name=runtime routine,
	description={a procedure provided by the runtime with a possibly different \gls{cc} and more privileges afforded to it than a normal procedure}
}

\makeglossaries