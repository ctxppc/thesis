\documentclass[main.tex]{subfiles}
\begin{document}
\onlyinsubfile{\mainmatter{}}

\chapter{Glyco}
Glyco\footnote{From the Greek term $\gamma\lambda\upsilon\kappa{}o$ meaning "sweet", alluding to the taste of a wild cherry.} is a compiler targeting the CHERI-RISC-V architecture, producing software running on CheriBSD systems and on a Sail-based emulator of the CHERI-RISC-V ISA.

Glyco is built following a \emph{nanopass} compiler design, described by several authors in the context of compiler education \cite{:educomp} but also commercial compiler development \cite{:commcomp}. A nanopass compiler consists of numerous small passes, hence the term \enquote{\nanopass{}}, which translate one \il{} to another. Source code is parsed into the compiler’s first \il{}, which is then translated via nanopasses through different \ils{}, ending up in the compiler's final \il{} (called Assembly, abbreviated S) that is simply CHERI-RISC-V assembly. Glyco then feeds this assembly to LLVM which builds \& links an executable ELF file. This last step is unrelated to the research problem, hence this dependency on LLVM.

Glyco currently defines 13 \ils{}. Each \il{} has a name and corresponding abbreviation, usually named after the abstraction it provides over the \lowerlang{}. Starting from the highest \il{}, these are:

\begin{itemize}
	\item EX (Expressions): A language that introduces expression semantics for values, thereby abstracting over computation effects.
	\item LS (Lexical Scopes): A language that introduces lexical scopes of definitions
	\item DF (Definitions): A language that introduces definitions with function-wide namespacing.
	\item CV (Computed Values): A language that allows computation to be attached to an assigned value.
	\item CA (Canonical Assignments): A language that groups all effects that write to a location under one canonical assignment effect.
	\item CC (Calling Convention): A language that introduces parameter passing and enforces the low-level Glyco calling convention.
	\item AL (Abstract Locations): A language that introduces abstract locations, i.e., locations whose physical locations are not specified by the programmer.
	\item CD (Conditionals): A language that introduces conditionals in effects and predicates, thereby abstracting over blocks (and jumps).
	\item PR (Predicates): A language that introduces predicates in branches.
	\item BB (Basic Blocks): A language that groups effects into blocks of effects where blocks can only be entered at a single entry point and exited at a single exit point.
	\item FO (Frame Operands): A language that introduces flexible operands in instructions, i.e., instructions that can take frame locations in all operand positions.
	\item FL (Frame Locations): A language that introduces frame locations, i.e., memory locations relative to the frame capability `cfp`.
	\item RV (CHERI-RISC-V): A language that maps directly to CHERI-RISC-V (pseudo-)instructions.
	\item S (CHERI-RISC-V Assembly): The ground language as provided to Clang for assembly and linking.
\end{itemize}

\biblio{}
\onlyinsubfile{\glsaddall\printglossaries}
\end{document}
